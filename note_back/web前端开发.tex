% Created 2025-12-04 四 17:29
% Intended LaTeX compiler: pdflatex
\documentclass[11pt]{article}
\usepackage[utf8]{inputenc}
\usepackage[T1]{fontenc}
\usepackage{graphicx}
\usepackage{longtable}
\usepackage{wrapfig}
\usepackage{rotating}
\usepackage[normalem]{ulem}
\usepackage{amsmath}
\usepackage{amssymb}
\usepackage{capt-of}
\usepackage{hyperref}
\date{\textit{<2022-01-24 一>}}
\title{Web前端开发}
\hypersetup{
 pdfauthor={CYQ},
 pdftitle={Web前端开发},
 pdfkeywords={},
 pdfsubject={},
 pdfcreator={Emacs 30.2 (Org mode 9.7.11)}, 
 pdflang={English}}
\begin{document}

\maketitle
\tableofcontents

\section{第 1 篇 HTML5}
\label{sec:org2c6eeeb}
\subsection{第1章 HTML5 学习概述}
\label{sec:orgcf06c5e}
HTML 是 HyperText Markup Language (超文本标记语言) 的缩写.
\subsubsection{本章概述}
\label{sec:org6e9f7ca}
\begin{enumerate}
\item 了解 HTML 的发展历程及影响
\item 安装 HTML 的开发软件
\item 掌握 HTML 的基本结构和语法
\end{enumerate}
\subsubsection{认识 HTML5}
\label{sec:org3c9334a}
HTML5 是 HTML 最新的修订版本, 2014 年 10 月由万维网联盟(W3C) 完成标准制定. 是跨平台的.
\begin{enumerate}
\item HTML 的发展历程
\label{sec:org943afd6}
通俗来讲,HTML 就是网页的源代码,任何一个网页都是由一行行HTML 代码编写而成的。

HTML 的第一个版本诞生于20 世纪七八十年代,当时互联网没有普及,也没有专业的组织制定HTML 规范。因此,那个时代HTML 的发展非常混乱,并没有受到开发者的重视,更没有得到大幅度的发展,HTML 还是一门冷门的语言。

HTML 真正崛起是从1998 年诞生的HTML4.0 版本开始的,紧接着在1999 年更新了HTML4.01 版本。自HTML4.01 版本以后,Web 世界经历了巨变。此时,被称为BAT 三巨头的百度、阿里巴巴、腾讯等互联网企业相继崛起,标志着互联网时代的到来。

HTML5 是由W3C(万维网联盟)于2007 年正式立项的,直至2014 年10 月底,这个长达八年的规范终于制定完成并公开发布。

HTML5 将会取代HTML4.01、XHTML1.0 标准,使网络标准满足互联网应用迅速发展的需求,为移动平台带来多媒体,推动Web 进入新的时代。
\item HTML5 与 HTML4 的区别
\label{sec:orge5b9422}
\begin{enumerate}
\item 语法简单化
更简单的doctype 是HTML5 中众多新特征之一。在HTML5 中,头部只需要写<!DOCTYPE html>即可。HTML5 的语法兼容HTML4 和XHTML1,但不兼容SGML。
\item 新增语义化标签
新增语义化标签新增加的语义化标签(如<header>、<footer>、<section>等)使得网页的可读性变得更高,也更加明确地表示出网页的结构,对于搜索引擎优化(SEO)有很大帮助。
\item 新的媒体标签
新的<audio>和<video>标签可以用来嵌入音频文件和视频文件。这些标签的使用让网页播放音频、视频更加方便。
\item 使用画布标签绘制图形
<canvas>标签具有绘图功能,通过与JavaScript 脚本的搭配,可在网页上绘制图像。
\item 新的表单设计
在 HTML5 中,表单增加了新元素,也为表单元素增加了许多新属性,让表单的使用更加便利。
\item 废除了一些旧标签
HTML5 废除了一些标签,其中大部分为网页美化标签,如<center>、<font>、<tt>、<big>、<dir>、<marquee>、<frame>等。
\end{enumerate}
\end{enumerate}
\subsubsection{学习 HTML5 前的准备工作}
\label{sec:org7539669}
\begin{enumerate}
\item 常用浏览器介绍
\label{sec:orge0558b3}
\begin{enumerate}
\item Internet Explorer
\item Google Chrome
\item Mozilla Firefox
\end{enumerate}
\item 浏览器内核介绍
\label{sec:org068afb5}
浏览器内核主要分成两部分:渲染引擎和JavaScript 引擎。
\begin{enumerate}
\item 渲染引擎负责获取网页内容(HTML、XML、图像等)、整理信息(如加入CSS等)以及计算网页的显示方式,然后输出至显示器或打印机。所有网页浏览器、电子邮件客户端及其他需要编辑、显示网络内容的应用程序都需要内核。浏览器内核的不同对于网页的语法解释也会有不同,所以渲染的效果也不同。
\item JavaScript 引擎负责解析和执行JavaScript 来实现网页的动态效果。
\end{enumerate}
开始,渲染引擎和JavaScript 引擎并没有区分得很明确,后来,JavaScript 引擎越来越独立,内核就倾向于指渲染引擎。

\textbf{常见的浏览器内核}
\begin{center}
\begin{tabular}{lll}
内核 & 浏览器 & 备注\\
Trident(IE 内核) & IE 浏览器、 360 浏览器、遨游浏览器、百度浏览器等 & 部分浏览器新版本是双核,一个内核是 Trident内核,再增加一个其它内核\\
Gecko 内核 & firefox & 代码完全开放\\
Presto & opera 早期版本内核 & 现在已经废弃\\
Webkit 内核 & Chrome、safari、Android 默认浏览器 & \\
\end{tabular}
\end{center}
\end{enumerate}
\subsubsection{HTML5 的语法与结构}
\label{sec:orgfa35a57}
HTML5 主要用标签来组织。
\begin{enumerate}
\item HTML5的语法
\label{sec:org306d4ac}
\begin{enumerate}
\item HTML5 标签
标签是HTML5 最基本单位和最重要的组成。使用“<”和“>”括起来,标签都是闭合的(规范)。标签分为单标记和双标记,单标记只有起始标记而没有结束标记,双标记是成对的开始标记和结束标记,基本语法如下:

\begin{verbatim}
<hr/> <!--单标记也叫自结束标记-->
<title></title> <!--标准标记,前面是开始标记,后面是结束标记,标记可以嵌套,但不能交叉嵌套-->
\end{verbatim}

 \textbf{html5标签是有相应的语意的。
*部分 HTML5 标签}
\begin{center}
\begin{tabular}{llll}
HTML5 标签 & 作用 & HTML5 标签 & 作用\\
html & 定义 html 文档 & body & 定义文档体 body\\
head & 定义文档头信息 & title & 定义文档的标题\\
a & html 链接标签 & img & html 图像标签\\
div & html 层 & table & 定义 html 表格\\
tr & 定义表格行 & td & 定义表格列\\
form & html 表单标签 & input & 定义表单的输入域\\
\end{tabular}
\end{center}

\item HTML5 标签属性
标签属性是标签的一部分,用于包含额外的信息。一个标签中可以有多个属性,并且属性和属性值成对出现,基本语法如下:

\begin{verbatim}
<img src=“../image/a.png” width=“100” height=“100”/>
<!-- 结构是 属性名=”属性值” -->
\end{verbatim}

\item HTML5 文档注释
注释是对文档的解释,浏览器不会对注释内容进行解析并呈现到页面上,只有查看HTML5 文件源代码时才会看到注释,基本语法如下:

\begin{quote}
<!-- 这是 HTML5 注释-->
\end{quote}
\end{enumerate}
\item HTML5的文档结构
\label{sec:org897f1c7}
HTML5 文件均以<html>标记开始,以</html>标记结束。一个完整HTML5 文件包含头部和主体两个部分的内容,在头部标记<head></head>里可以定义标题、样式等,文档的主体<body></body>中的内容就是浏览器要显示的信息。
HTML4.01 之前的文档声明,语法结构如下:

\begin{verbatim}
<!DOCTYPE html PUBLIC "-//W3C//DTD HTML 4.01//EN"
"http://www.w3.org/TR/html4/strict.dtd">
\end{verbatim}

HTML5已经对文档声明进行了简化,语法结构如下

\begin{quote}
<!DOCTYPE html>
\end{quote}

HTML5 文档的基本结构, 代码如下:
\begin{verbatim}
      <!DOCTYPE html>
<html>
  <head>
    <meta charset="utf-8" />
    <title>我的第一个网页</title>
  </head>
  <body>
    Hello World!
  </body>
</html>
\end{verbatim}

\textbf{注意:页面中必须有文档声明,而且必须在文档页面的第一行!}
\begin{enumerate}
\item 头部内容<head>
\label{sec:org3b32e39}
<head>标签用于表示网页的元数据,即描述网页的基本信息。其中主要包含以下标签:
1)<title>标签用于定义页面的标题,是成对标记,位于<head>标签之间。
2)<meta>标签用于定义页面的相关信息,为非成对标记,位于<head>标签之间,<meta>标签可以描述页面的作者、摘要、关键词、版权、自动刷新等页面信息。
\begin{enumerate}
\item <link>标签,加载一个图片作为网页图标。
\end{enumerate}
\begin{enumerate}
\item <meta> 标签的常用属性
\label{sec:orgcecfd4c}
\begin{enumerate}
\item charset 属性:设置文档的字符集编码格式
基本语法如下:
\begin{verbatim}
<mtea charset="UTF-8">
\end{verbatim}

HTML4.01 之前的文档设置字符集编码,基本语法如下:
\begin{verbatim}
<meta http-equiv="Content-Type" content="text/html; charset=UTF-8" /> 
\end{verbatim}

常见的字符集编码格式包括GB2312、GBK、UTF-8 等。
GB2312 是国标码,简体中文。GBK 是扩展的国标码。UTF-8 是一种针对Unicode 的可变长度字符编码,也称万国码(常用)。

\item http-equiv 属性:将信息写给浏览器看,让浏览器按照这里面的要求执行,可选属性值有Content-Type(文档类型)、refresh(网页定时刷新)、set-cookie(设置浏览器cookie 缓存),需要配合content 属性使用。http-equiv 属性只是表明需要设置哪一部分,具体的设置内容需要放到content 属性中。

\item name属性:将信息写给搜索引擎看。 使用方法同 http-equiv 属性
常用的属性值有author(作者)、keywords(网页关键字)、description(网页描述),它们在网页中必不可少。
基本语法如下:
\begin{verbatim}
 <!--作者-->
 <meta name="author" content="http://www.jredu100.com" />
<!--网页关键字:多个关键字用英文逗号分隔-->
 <meta name="keywords" content="HTML5,网页,Web 前端开发" />
 <!--网页描述:搜索网站时,title 下面的解释文字。-->
 <meta name="description" content="这是我在杰瑞教育开发的第一个网页。" />
\end{verbatim}
\end{enumerate}
n**** <link>标签的常用属性 
       使用<link>标签可以加载一个图片作为网页图标。<link>标签有如下属性:
       1)rel 属性:声明被链接文件与当前文件的关系,此处选icon。 可以理解为类型
       2)type 属性:声明被链接文件的类型,可以省略。
       3)href 属性:表示图片的路径地址。

基本语法如下:
\begin{verbatim}
<link rel="icon" type="image/x-icon" href="img/icon.jpg" />
\end{verbatim}
\end{enumerate}
\item 主体内容
\label{sec:orga810940}
标记<body></body>包含文档所有的内容,如文字、图像、表格、表单等元素。例如,在<body>中使用语义化标记设计网页,基本语法如下:
\begin{verbatim}
  <body>
    <header>网站主题</header>
    <nav>连接菜单</nav>
    <article>
      主内容
      <section>
        章节段落
      </section>
   </article>
   <aside>侧边栏</aside>
   <footer>页脚</footer>
</body>
\end{verbatim}
\end{enumerate}
\end{enumerate}
\subsubsection{章节案例: 开始我的第一个网页}
\label{sec:org041d6f5}
\begin{verbatim}
<!DOCTYPE html>
<html>
  <head>
    <meta charset="utf-8" />
    <meta name="keywords" content="杰瑞教育,HTML5,网页开发" />
    <meta name="description" content="这是我开发的第一个网页!" />
    <title>我的第一个网页</title>
    <link rel="icon" href="img/icon.jpg"/>
  </head>
  <body>
    欢迎来到 HTML5 的奇幻世界!
  </body>
</html>
\end{verbatim}
\subsubsection{第 2 章 HTML5 常见的块级标签和行级标签}
\label{sec:org5dae75a}
\begin{enumerate}
\item 本章目标
\label{sec:org548408b}
\begin{enumerate}
\item 掌握常见的块级标签
\item 掌握常见的行级标签
\item 掌握行级标签和块级标签的区别
\item 了解 HTML5 的新增标签
\end{enumerate}
\item 常见的块级标签
\label{sec:orgd6800ae}
块级标签,顾名思义,此类标签在网页中显示为块。块级标签一般独占一行,新的块级标签将从新的一行开始排列,它可以容纳内联元素和其他块级元素。
\begin{enumerate}
\item 标题标签
\label{sec:org7b66064}
标题标签的作用是设置段落标题的大小,共设置了6 级,从1 级到6 级每级标题的字体大小依次递减。
基本语法如下:
\begin{verbatim}
<h1>标题文字</h1>
\end{verbatim}
\item 水平线标签 hr
\label{sec:orgdb6c935}
水平线标签的作用是添加水平分隔线,让页面更容易区分段落。
基本语法如下:
\begin{verbatim}
<hr />
\end{verbatim}
\item 段落标签 p
\label{sec:org7b4f80c}
使用段落标签可以区分段落,不同的段落间会自动增加换行符,段落上下方各会有一个空行的间隔。
基本语法如下:
\begin{verbatim}
<p>段落文字</p>
\end{verbatim}
\item 换行标签br
\label{sec:org57b95ff}
使用换行标签可以控制段落中文字的换行显示。一般的浏览器会根据窗口的宽度自动将
文本进行换行显示。 基本语法如下:
\begin{verbatim}
<br />
\end{verbatim}
\item 引用标签 blockquote
\label{sec:org1ad864d}
使用引用标签来表示引用的文字,同时会将标签内的文字缩进显示。其cite 属性表明引用的来源,一般表明引用网址。
基本语法如下:
\begin{verbatim}
<blockquote cite=" http://www.jredu100.com">引用的文字</blockquote>
\end{verbatim}
cite属性可以省略, 引用有缩进效果。
\item 预格式标签 pre
\label{sec:org327ef15}
预格式标签可以将文字按照原始的排列方式进行显示(保留原有的换行和缩进)
基本语法如下:
\begin{verbatim}
<pre>需要按原
      格式显示的
           文字
</pre>
\end{verbatim}
\item <ul><li></li></ul>:无序列表标签
\label{sec:org691563e}
无序列表是将文字段落向内缩进,并在每个列表项前面加上圆形(●)、空心圆形(○)或方形(■)等符号,以达到醒目的效果。由于无序列表没有顺序编号,而是采用项目符号的形式,所以又被称为符号列表。 基本语法如下:
\begin{verbatim}
<ul type="disc">
  <li>第一项</li>
  <li>第二项</li>
  <li>第三项</li>
</ul>
\end{verbatim}
\item <ol><li></li></ol>:有序列表标签
\label{sec:orgf27c94e}
基本语法如下:
\begin{verbatim}
<ol type="1">
  <li>第一项</li>
  <li>第二项</li>
  <li>第三项</li>
</ol>
\end{verbatim}
\end{enumerate}
\end{enumerate}
\subsubsection{第 3 章 HTML5 表格}
\label{sec:orgbde22c9}
\begin{enumerate}
\item 本章概述
\label{sec:org652466f}
 了解表格的基本结构。
 掌握表格的基本属性。
 掌握表格的行、列的基本属性。
 了解表格的结构化和直列化。
\item html5 表格简介
\label{sec:orgd565d34}
\begin{enumerate}
\item 表格的基本结构
\label{sec:org79e42fe}
表格的基本结构由行列组成,单元格是表格的最基本单位。
\item 表格的定义
\label{sec:org90e11ce}
表格由<table> 标签定义。每个表格均有若干行,行由<tr> 标签定义,每行被分割为若干单元格,列由<td> 标签定义。字母td 指表格数据(table data),即数据单元格的内容。如果表格的第一行为表头,那么<td>标签需要用<th>标签替换掉。数据单元格可以包含文本、图片、列表、段落、表单、水平线、表格等。
\begin{verbatim}
  <!DOCTYPE html>

<html>
  <head>
    <meta charset="utf-8" />
    <title>表格的基本结构</title>
  </head>
  <body>
    <table>
      <tr>
        <th>表头1</th>
        <th>表头2</th>
        <th>表头3</th>
      </tr>
      <tr>
        <td>第一行1</td>
        <td>第一行2</td>
        <td>第一行3</td>
      </tr>
      <tr>
        <td>第二行1</td>
        <td>第二行2</td>
        <td>第二行3</td>
      </tr>
    </table>
  </body>
</html>
\end{verbatim}
\end{enumerate}
\item 表格的基本属性
\label{sec:org33de558}
表格的属性可以分为两大类,分别为作用于<table>标签和作用于行<tr>、列<td>的属性。
\begin{enumerate}
\item border:表格边框宽度
\label{sec:org883c1d9}
border 属性定义表格边框,属性值可以接收整数类型的数字,表示设置表格的宽度。基本语法如下:
\begin{verbatim}
<table border="1"> </table>
\end{verbatim}

\begin{quote}
注意:如果border 的值增大,则只有表格最外围框线增加,每个单元格上的边框并不会变化。表格的border=5 时的显示效果如图3-6 所示。
\end{quote}
\item width/heigth: 宽度/高度属性
\label{sec:org5ebd5bb}
基本语法如下:
\begin{verbatim}
<table border="1" width="400" height="200"> </table>
\end{verbatim}
\item bgcolor:背景色属性
\label{sec:orgc9c2ee7}
基本语法如下:
\begin{verbatim}
<table border="1" bgcolor="red"> </table>
\end{verbatim}
\item background:表格背景图属性
\label{sec:org3a6fd4f}
background 属性定义表格的背景图,需要传入一张图片的路径地址。当background 背景图属性与bgcolor 背景色属性同时存在时,背景图会覆盖掉背景色。

基本语法如下:
\begin{verbatim}
<table border="1" background="img/img.png"> </table>
\end{verbatim}
\item bordercolor : 边框颜色属性
\label{sec:org2b9ae36}
基本语法如下:
\item cellspacing:单元格间距属性
\label{sec:org3d69db9}
定义单元格和单元格之间的距离。 取值 0 时, 表示单元格之间没有间距。
注意:cellspacing 取值为 0, 不能合并掉边框的宽度, 边框的宽度还是存在的。 
\item cellpadding:单元格内边距属性
\label{sec:org61b2c73}
单元格中的文字与单元格边框之间的距离。

基本语法如下:
\item align: 表格对齐属性
\label{sec:org3bc142b}
用于调整表格在其父容器中的位置,可选值有left、center、right,分别表示表
格居左、居中、居右。

基本语法如下:
注意:表格的align 属性现在不再建议使用。它的相关功能已经被CSS 中的浮动(float)所取代
\end{enumerate}
\item 行和列的属性(tr、td 的属性)
\label{sec:orgc59f117}
\begin{enumerate}
\item width/height:单元格宽度/高度属性
\label{sec:org83d3c4f}
主要用于调整表格中单元格的宽和高。
\begin{verbatim}
<table border="1">
  <tr>
    <td width="100" height="50">第一行1</td>
    <td>第一行2</td>
    <td>第一行3</td>
  </tr>
</table>
\end{verbatim}

\textbf{注意:}  当表格属性与行列属性冲突时,以行列属性为准。
\item bgcolor: 单元格背景色属性
\label{sec:orgac47635}
语法如下:
\begin{verbatim}
<table border="1">
  <tr>
    <td bgcolor="#0000FF">第一行 1</td> <td>第一行 2</td>
    <td>第一行 3</td>
</tr> </table>
\end{verbatim}
\item align: 单元格内容水平对齐属性
\label{sec:orga0c48be}
语法如下:
\begin{verbatim}
<table border="1" width="300" height="50">
<tr>
   <td align="left">第一行 1</td>
   <td align="center">第二行 2</td>
   <td align="right">第三行 3</td>
</tr>
</table>
\end{verbatim}

\textbf{注意:}
\begin{quote}
表格的align属性是控制表格自身在浏览器中的显示位置,而行列的 align 属性是控制单 元格中文字在单元格中的对齐方式。

表格的 align 属性并不影响表格内文字的水平方式,<tr>标签的 align 属性可以控制一行 中所有单元格的水平对其方式。
\end{quote}
\item valign:单元格内容垂直对齐属性
\label{sec:orgcc95bba}
语法如下:
\begin{verbatim}
<table border="1" width="200" height="100">
  <tr>
    <td valign="top">第一行 1</td>
    <td valign="center">第一行 2</td>
    <td valign="bottom">第一行 3</td>
  </tr>
</table>
\end{verbatim}

\textbf{注意:} 当表格属性与行列属性冲突时,以行列属性为准(近者优先)。
\item colspan/rowspan: 表格的扩行于跨列
\label{sec:org765f123}
colspan 属性表示跨列,当某个格跨 N 列时,其右边 N-1 个单元格需删除。
rowspan 属性表示跨行,当某个格跨 N 行时,其下方 N-1 个单元格需删除。

语法如下:
\begin{verbatim}
<table border="1">
<tr>
   <td colspan="3">学生成绩</td>
</tr>
<tr>
   <td rowspan="2">张三</td>
   <td>语文</td>
   <td>98</td>
</tr>
<tr>
   <td>数学</td>
   <td>95</td>
</tr>
<tr>
   <td rowspan="2">李四</td>
   <td>语文</td>
   <td>88</td>
</tr>
<tr>
   <td>数学</td>
   <td>91</td>
</tr>
</table>
\end{verbatim}
\end{enumerate}
\item 表格的结构化与直列化
\label{sec:org251f7ce}
为了更好地管理及格式化表格,更好地语义化标签,需要掌握表格的结构化与直列化。
\begin{enumerate}
\item 表格的结构化
\label{sec:orga0e01a1}
表格的结构化就是将表格分为表头、表体、表尾三部分,这样在修改其中一部分时不会影响到其他部分,方便对表格进行操作。

一个完整的表格通常包括以下四部分:
\begin{enumerate}
\item caption:表格的标题,通常出现在表格的顶部。
\item thead:定义表格表头,通常表现为标题行。
\item tbody:定义一段表格主体,一个表格可以有多个主体,可以按照行来进行分组。
\item tfoot:定义表格的脚尾,通常表现为总计行。
\end{enumerate}

基本语法如下:
\begin{verbatim}
<table width="500">
  <caption>表格标题</caption>
  <thead>
    <tr>
      <th>表格头部</th>
    </tr>
  </thead>
  <tbody>
    <tr>
      <td>表格主体</td>
    </tr>
  </tbody>
  <tfoot>
    <tr>
      <td>表格底部</td>
    </tr>
  </tfoot>
</table>
\end{verbatim}

tbody 包含行的内容下载完优先显示,不必等待表格结束。另外,还需要注意表格行本来是从上向下显示的,但是应用了<thead><tbody><tfoot>以后,就“从头到脚”显示,即不 管行代码顺序如何,即使<thead>写在了<tbody>的后面,网页显示时,还是先<thead>后 <tbody>显示。
\item 表格的直列化
\label{sec:org79a2691}
通过设置表格的直列化可以对表格的列进行分组,以便对其进行格式化。 基本语法如下:

\begin{verbatim}
<table width="500">
  <colgroup style="background-color: yellow;"> <!--前两列为一组-->
        <col /> <!--第一列-->
        <col /> <!--第二列--> </colgroup>

    <col style="background-color: blue;"/> <!--第三列-->
      <caption>表格标题</caption>
    <thead>
      <tr>
        <th>头 1</th>
        <th>头 2</th>
        <th>头 3</th> </tr>
    </thead>
    <tbody>
      <tr>
        <td>111</td>
        <td>111</td>
        <td>111</td> </tr>
      <tr>
        <td>222</td>
        <td>222</td>
        <td>222</td>
      </tr>
    </tbody>
    <tfoot>
      <tr>
        <td>尾 1</td>
        <td>尾 2</td>
        <td>尾 3</td> </tr>
   </tfoot>
</table>
\end{verbatim}

如需对全部列应用样式,<colgroup> 标签很有用,这样就不需要对各个单元和各行重复应用样式了。

注意:<colgroup> 标签只能在<table>中使用。
\end{enumerate}
\end{enumerate}
\subsubsection{第 4 章 HTML5 表单}
\label{sec:org1117804}
\begin{enumerate}
\item 本章目标
\label{sec:org94682c9}
\begin{itemize}
\item 熟悉表单的结构。
\item 掌握表单的 input 元素及其他元素。
\item 掌握表单 input 元素的 type 属性。
\item 了解 HTML5 智能表单的新增元素及属性。
\end{itemize}
\item 表单的结构
\label{sec:org07c1722}
表单由许多表单控件组成,主要包括用户填写信息部分和表单提交按钮。

基本语法如下:
\item 表单的常用属性
\label{sec:org8a947e8}
表单的常用属性有 3 种
\begin{enumerate}
\item action 属性
action 属性用于指定表单提交时向何处发送表单数据,即需要发送的服务器地址。基本。
\end{enumerate}

语法如下:
\begin{enumerate}
\item method 属性
method 属性用于指定表单向服务器提交数据的方法,包括两种方法,分别是 get 和post。这两种方法各有特点,下面分别进行具体介绍。
(1) get方法
使用 URL(统一资源定位符)传递参数:\url{http://服务器地址?name1=value1\&name2}=
value2,其中“?”符号表示要进行参数传递,“?”符号后面采用“name=value”的形式传 递,多个参数之间,用“\&”符号连接。URL 传递的数据量有限,只能传递少量数据。

注意:使用 URL 传递参数并不安全,所有信息可在地址栏中看到,并且可以通过地址 栏随意传递其他数据。

(2)post 方法
 将数据封装后使用 http 请求传递,信息在地址栏中不可见,比较安全,并且传递的数据量理论上没有限制。

综上所述,虽然 get 方法是表单提交的默认方法,但是通常采用 post 方法传递数据。

基本语法:

\item enctype 属性
enctype 属性指定表单发送的编码方式,只有 method="post"时才有效,共有三种属性值。
(1)  application/x-www-form-urlencoded:此为默认值,如果 enctype 属       性省略不写,则表示采取此种编码方式。
(2)  multipart/form-data:不对字符编码,用于上传文件时使用。
(3)  text/plain:将空格转换为“+”符号,但不编码特殊字符,通常在将表单发送到      电子 邮箱时使用。
\end{enumerate}
\item input 输入框
\label{sec:org39d4b30}
作为表单最重要的元素,input 输入框用于搜集用户信息。根据不同的 type 属性值,可 以用多种形式输入内容。例如,当 type 值为 password 时就可以设置输入框为输入密码形 式,此时用户输入的内容用小黑点代替显示。灵活使用 input 输入框可以让表单收集更多的 信息,下面来具体学习 input 输入框的相关属性和类型。
\begin{enumerate}
\item input 常用属性
\label{sec:org0cc0dd8}
\begin{enumerate}
\item type 属性
type 属性表示 input 输入框的类型,它的默认值是 text。所有浏览器都支持 type 属性, 但是 type 的属性值并不是所有浏览器都可以支持,本节介绍的属性值所有浏览器均可支持, 但后续小节提到的某些 HTML5 表单新增属性值则需要注意浏览器的兼容性。
\item name 属性
name 属性表示 input 输入框的名称,一般必填。因为传递数据时,使用“name=value” 的形式传递。
\item value 属性
value 属性表示 input 输入框的默认值。

代码示例:
\begin{verbatim}
<form action="form.php" method="post">
   请输入内容:
   <input name="text1" type="text" value="输入框的默认值"/>
</form>
\end{verbatim}

\item placeholder 属性
placeholder 属性表示输入框中的提示信息,当输入框有 value 属性的时候,提示内容会 消失,转而显示 value 属性值。

\item tabindex属性
abindex="1" 此属性控制按 Tab 键时的跳转顺序,从最小的数值开始,逐个往大的数值 跳转,依次获得焦点。

\item input元素的其他属性
\begin{enumerate}
\item checked="checked" 默认选中。
\item disabled="disabled" 设置控件不能使用。用在按钮上效果为不能单击,用在输入框上
\end{enumerate}
\end{enumerate}
则表示不能修改。而且,如果输入框设置为 disabled,则输入框中的信息不能往后台传递。
\begin{enumerate}
\item hidden="hidden" 设置隐藏控件。基本语法如下:
\begin{verbatim}
<input type="hidden" name="username" value="1234" />
\end{verbatim}
\end{enumerate}
\item text: 文本输入框
\label{sec:orgac507eb}
文本输入框用于输入单行文本,默认宽度为 20 个字符。在登录注册时,常常用到文本 输入框,它主要用于填写用户名称。

代码示例如下:
\begin{verbatim}
  <form action="form.php" method="post">
请输入内容:
<input name="text1" type="text" maxlength="10" size ="5"/>
<!--上述代码表示这个文本输入框的最大字符长度为 10,可显示的字符数为 5 --> </form>
\end{verbatim}
\item passowrd: 密码输入框
\label{sec:org5f70da3}
密码输入框输入的内容会以小黑点的形式替代显示。
代码示例如下:
\begin{verbatim}
  <form action="form.php" method="post">
请输入内容:
<input name="pwd" type="password" maxlength="16"/> </form>
\end{verbatim}
\item radio: 单选按钮
\label{sec:orgdfac7d2}
代码示例如下:
\begin{verbatim}
<form action="form.php" method="post">
  <input type="radio" name="sex" value="男" checked="checked" />男 <input type="radio" name="sex" value="女" />女
</form>
\end{verbatim}

注意:
\begin{itemize}
\item name 和 value 属性需同时存在,提交时,提交的是 value 中的属性值。 例如:<input type="radio" name="sex" value="男"/> 提交时,显示"sex=男"。
\item radio 凭借 name 属性区分是否为同一组。name 相同,为同组,同组中只能选择一个。
\item checked="checked" 默认选中。radio 只能选一个,checkbox 可以选多个。
\end{itemize}
\item checkbox: 复选按钮
\label{sec:org9719be4}
代码示例如下:
\begin{verbatim}
  <form action="form.php" method="post">
    爱好选择:
    <input type="checkbox" name="hobby" value="sing" checked="checked" />唱歌 <input type="checkbox" name="hobby" value="draw" checked="checked" />画画 <input type="checkbox" name="hobby" value="dance" />跳舞
</form>
\end{verbatim}
\item file: 文件上传按钮
\label{sec:orga190c70}
文件上传按钮用于文件上传,单击“选择文件”按钮会弹出对话框,选择需要上传的文件。
\item submit: 表单提交按钮
\label{sec:orgef1145b}
表单提交按钮用于提交表单数据,单击按钮后,表单中用户填写的信息会被发送到表单 指定的地方进行处理。图 4-10 为一个设置了 value 值的 submit 表单提交按钮。当没有 value 值时,submit 按钮中默认显示的文字为“提交”。

代码示例如下:
\begin{verbatim}
  <form action="form.php" method="post">
    <input type="submit"" value="登录"/>
</form>
\end{verbatim}
\item reset : 重置按钮
\label{sec:org9c036bc}
重置按钮将表单数据重置为初始状态,通常是清空表单已填内容。
代码示例如下:
\begin{verbatim}
<form action="form.php" method="post">
  <input type="text" value="输入框的默认值"/>
  <input type="text" placeholder="请输入"/>
  <input type="reset"/>
</form>
\end{verbatim}
\item image: 图形提交按钮
\label{sec:org973e5cd}
图形提交按钮需要添加 src 属性来设置图片路径。功能与 submit 相同,可以提交表单数据,通常在美化网页时会用到图形提交按钮来代替默认的提交按钮,使页面更加美观。
示例代码如下:
\begin{verbatim}
<form action="form.php" method="post">
  <input type="image" src="http://www.jredu100.com/statics/images/index/top/logo.png"/>
</form>
\end{verbatim}
\item button : 可单击按钮
\label{sec:org626ecb4}
定义一个可单击的按钮,通常与 JavaScript(后面会有专门的篇章讲解)一起使用来启 动脚本。下面的代码就利用 button 按钮在浏览器中显示了一个弹窗,图 4-14 是单击“点 我!”按钮后出现弹窗的效果。
代码示例如下:
\begin{verbatim}
<form action="form.php" method="post">
  <input type="button" value="点我!" onclick="alert('这是一个按钮!')" />
</form>
\end{verbatim}
\end{enumerate}
\item 其它表单元素
\label{sec:orgd3dfcf8}
\begin{enumerate}
\item select下拉选择控件
\label{sec:org4db76c4}
代码示例如下:
\begin{verbatim}
<form action="form.php" method="post">
  <select name="week">
    <option value="1">1</option>
    <option value="2">2</option>
    <option value="3">3</option>
    <option value="4">4</option>
    <option value="5">5</option>
    <option value="6">6</option>
    <option value="7">7</option>
  </select>
</form>
\end{verbatim}
\end{enumerate}
\item HTML5智能表单
\label{sec:org52dfee6}
在 HTML5 中,表单新增了一些属性和元素,这些属性和元素让表单变得更加方便实 用。例如,autocomplete 属性可以让表单具有自动完成功能,浏览器会根据用户之前输入的 值自动完成,这就让表单的填写更加方便。
\begin{enumerate}
\item 表单分组<fieldset>
\label{sec:org087fe9e}
基本语法如下:
\begin{verbatim}
<form action="form.php" method="post">
  <fieldset >
    <legend>这是一个表单</legend>
    其他表单控件
  </fieldset>
</form>
\end{verbatim}

其中,<fieldset >表示表单边框,<legend>表示表单标题。如果想要让标题嵌入到边框 中,则需将标题标签写到边框标签里面,就像上面代码示例中所写的一样。另外,一个表单 可以有多个边框与标题的组合。
代码示例如下:
\begin{verbatim}
<form action="form.php" method="post">
  <fieldset >
    <legend>这是表单的第一部分</legend>
    其他表单控件
  </fieldset>
  <fieldset >
    <legend>这是表单的第二部分</legend>
    其他表单控件
  </fieldset>
</form>
\end{verbatim}
\item html5表单新增元素及属性
\label{sec:org185c2c9}
\begin{enumerate}
\item html5表单新增元素
\begin{center}
\begin{tabular}{ll}
新增元素 & 描述\\
<datalist> & <input>标签定义选项列表。它与<input>元素配合使用来定义<input>可能的值\\
<keygen> & <keygen> 标签规定用于表单的密钥对生成器字段\\
<output> & <output> 标签定义不同类型的输出,比如脚本的输出\\
\end{tabular}
\end{center}
\end{enumerate}
<datalist>具有和 autocomplete 类似的自动完成功能,但它还有一个功能是 autocomplete 属性所没有的,那就是在使用<datalist>时,它可以根据用户输入的内容,在预先设置好的列 表中筛选出与用户输入相关的信息作为备选。
基本语法如下:
\begin{verbatim}
<form action="form.php" method="post">
  <input type="text" list="list" />
  <datalist id="list">
    <option>123</option>
  </datalist>
</form>
\end{verbatim}

代码示例如下:
\begin{verbatim}
<form action="form.php" method="post">
  请输入:
  <input type="text" list="list" />
  <datalist id="list">
    <option>123</option>
    <option>abc</option>
    <option>456</option>
    <option>def</option>
    <option>789</option>
  </datalist>
</form>
\end{verbatim}

\begin{enumerate}
\item html5表单及其控件部分新增属性
(1)表单新增属性
\begin{center}
\begin{tabular}{ll}
属性 & 说明\\
autocomplete & 规定 form 表单具有自动完成功能。当用户在自动完成域中开始输入时,浏览器应该在该域中显示填写的 选项\\
novalidate & 规定在提交表单时不进行验证\\
\end{tabular}
\end{center}
\end{enumerate}
autocomplete 属性值有 on 和 off,novalidate 属性值有 true 和 false。

(2)<input>标签新增属性
\begin{center}
\begin{tabular}{ll}
属性 & 说明\\
autocomplete & 规定<input>标签具有自动完成功能\\
autofocus & 规定在页面加载时,控件自动地获得焦点(不过一个页面只能有一个控件使用该属性)\\
required & 规定输入的字段是必需的(必须填写)\\
pattem & 规定通过其检查输入值的正则表达式\\
form overrides & 规定表单重写属性\\
form & 规定输入域所属的一个或多个表单\\
\end{tabular}
\end{center}
(3)<input>标签新增输入类型
\begin{center}
\begin{tabular}{ll}
类型 & 作用\\
color & 拾色器\\
date & 输入时间格式\\
email & 输入email格式\\
number & 输入数字格式\\
range & 定义包含一定范围内的值的数字字段\\
search & 定义用于输入搜索字符串的文本字段\\
\end{tabular}
\end{center}
\end{enumerate}
\end{enumerate}
\section{第 2 篇 CSS3}
\label{sec:org501d453}
\subsection{第 5 章 CSS 基础知识}
\label{sec:org6c30d37}
\subsubsection{本章学习目标}
\label{sec:orged11f3e}
\begin{itemize}
\item 了解CSS的基本概念及语法结构
\item 了解页面中使用CSS的三种方式
\item 熟练掌握各种CSS选择器的使用
\item 了解CSS选择器命名规范及优先级
\end{itemize}
\subsubsection{CSS概述}
\label{sec:orgffefbea}
\begin{enumerate}
\item CSS简介
\label{sec:orgfa1a99a}
CSS 于 1996 年由 W3C 组织制定,最新的版本为 CSS3,主要用于美化网页。CSS 是对页面内容数据和显示风格分离思想的一种体现,通过建立定义样式的 CSS 文件,让 HTML 文件调用 CSS 文件所定义的样式,如果需要修改 HTML 中的部分显示风格,只要修改对应 的 CSS 文件即可,极大地提高了工作效率。
\item CSS语法结构
\label{sec:orgaacaaa7}
CSS 由两部分组成:选择器及一条或多条声明。选择器用于选中用户需要改变样式的 HTML 元素,选择器的声明部分由大括号包裹,每条声明由一个属性和一个属性值组成。属 性是需要对元素进行设置的样式属性,属性和属性值用冒号分开,多个属性之间用分号分隔。
基本语法如下:
\begin{verbatim}
选择器{ 属性:属性值;
        [属性:属性值; ...]
 }

h1{
    color:red;
}
\end{verbatim}

代码解释如下:

\begin{enumerate}
\item h1:选择器,表示要选择所有的 h1 标签。
\item color:属性名,表示要对字体颜色属性进行设置。
\item red:属性值,表示要设置字体颜色为红色。
\item 属性与属性值组成了一个声明,属性与属性值之间用冒号分隔。
\end{enumerate}

使用 CSS 时,注意事项如下:

\begin{enumerate}
\item CSS 是大小写不敏感的,但规范的写法是全部小写。
\item 规范性要求,每一行只写一个声明。
\item 规范性要求,每个声明后边需要添加分号作为结束符。
\item 所有符号均为英文,切勿使用中文符号。
\item 注意代码的缩进,用 HBuider 编写代码会有提示,避免拼写错误。
\end{enumerate}
\item CSS注释
\label{sec:orgc603c53}
为样式表添加注释有助于标记样式的作用范围以及复杂样式的作用等,便于进行后期的 维护。CSS 添加注释的方式为/*\ldots{}\ldots{}*/。注释代码示例如下:
\begin{verbatim}
  /*设置 h1 标签的样式*/
h1{
    /*设置字体颜色为红色*/
    color:red;
}
\end{verbatim}
\item 行内样式表
\label{sec:orgc161037}
行内样式表,顾名思义,就是将 CSS 代码放置在 HTML 代码内部,作为 HTML 标签的 属性存在,HTML 代码与 CSS 代码处于同一行中。

代码示例如下:
\begin{verbatim}
<a href="#" style="color:red;font-size:10px;">日用百货</a>
<!--产生一个红色的,字号是 10px 的超链接-->
\end{verbatim}

行内样式表的特点如下:
\begin{enumerate}
\item 将 CSS 代码与 HTML 代码糅合在一起,不符合 W3C 关于“内容与表现分离”的基 本规范,不利于后期维护。
\end{enumerate}
2 )可以单独定义某个元素的样式,灵活方便。
\begin{enumerate}
\item 优先级最高,但是不推荐使用,仅在测试时可以采用。
\end{enumerate}
\item 内部样式表
\label{sec:org70d8c4c}
内部样式表也称为内嵌样式表,是指 CSS 代码内嵌到 HTML 代码中,二者处于同一个 文件中,通常 CSS 代码放置在 HTML 代码的<head>标签内部。

代码示例如下:
\begin{verbatim}
 <head>
<!--charset="UTF-8"表示当前文档采用字符集中 utf-8,支持中文-->
<meta charset="UTF-8">
<title>内部样式表</title>
<!--内部样式表 代码要放置在 style 标签内-->
<style type="text/css">
  div{ color:red;
  }
</style>
</head>
\end{verbatim}

内部样式表的特点如下:
\begin{enumerate}
\item 写在<head>标签中,一定程度地将 CSS 代码与 HTML 代码进行了分离,但是分离不 够彻底,无法应用于其他 HTML 文件,实现样式复用。
\item 优先级低于行内样式表。
\end{enumerate}
\item 外部样式表
\label{sec:org92b4c7c}
外部样式表是指 CSS 代码完全独立出来,单独放置在扩展名为.css 的文件中,在 HTML 文件中,将 CSS 文件引入进来,形成关联。

代码示例如下:
\begin{verbatim}
 <head>
   <meta charset="UTF-8">
   <title>外部样式表</title>
   <link rel="stylesheet" type="text/css" href="css/ch05.css" />
</head>
\end{verbatim}

其中,<link>标签具有以下属性:
\begin{enumerate}
\item rel 属性:声明被链接文档与当前文档的关系,必写。
\item type 属性:被链接文档的类型,可写。
\item href 属性:被链接文档的地址,必写。
\end{enumerate}

外部样式表的特点如下:
\begin{enumerate}
\item 与内部样式表一样,写在<head>标签中,实现了 CSS 代码与 HTML 代码的彻底分
\end{enumerate}
离,方便样式复用与后期维护,符合 W3C 规范。
\begin{enumerate}
\item 优先级要低于内部样式表。
\end{enumerate}
3)后续开发中推荐使用此种方式。
\item CSS选择器
\label{sec:orgf83e0eb}
\begin{enumerate}
\item 通用选择器 *\{ \}
\label{sec:org6392bfd}

示例代码如下:
\begin{verbatim}
*{
padding: 0px;
margin: 0px;
font-family: "微软雅黑",sans-serif;
font-size: 12px;
}
\end{verbatim}
\item 标签选择器
\label{sec:orgaaa59ac}
\begin{itemize}
\item \textbf{写法} :HT-ML 标签名 \{ 样式属性:样式属性值;……\}
\item \textbf{优先级} :高于通用选择器
\item \textbf{示例代码如下:}
\end{itemize}
\begin{verbatim}
 div{
     width: 100%;
     height: 90px;
     background-color: red;
 }
/*HTML 部分代码*/
<div>这是一个 div</div>
\end{verbatim}
\item 类选择器
\label{sec:org9fef6ae}
\begin{itemize}
\item 写法: .类名称\{ \}
\item 调用:在需要改变样式的标签上,使用 \uline{class=“选择器名称”}  调用对应选择器。
\item 作用:修改所有调用选择器的标签。
\item 优先级:高于标签选择器。
\item 示例代码如下:
\begin{verbatim}
.first{
    width: 200px;
    color: #F00;
}
/*HTML body 部分代码*/
<div>
<ul>
<li class="first">家用电器</li>
<li>洗衣机</li>
……
</ul>
</div>
\end{verbatim}
\end{itemize}
代码含义:选择类名称为 first 的标签,并给标签设置宽度、字体颜色等样式属性。
注意事项如下:
1) 类名称是可以随意取名的,但通用做法是只能包含字母、数字、下画线,并且不以数字开头,否则可能会产生样式不能应用的问题。
2) 类名称应该能表示一定意义,不能起毫无意义的名字,如 a。
3) 当页面需要对多个元素应用相同样式,则采用类选择器。
4) 类选择器可以应用不同标签。 
\item id选择器
\label{sec:org8499143}
\begin{itemize}
\item 写法: \#id名称 \{ \}
\item 作用:在需要改变样式的标签上,使用 \uline{id="选择器名称"}  调用对应选择器。
\item \textbf{优先级:大于类选择器。}
\item 代码示例如下:
\begin{verbatim}
#list{
     width: 200px;
     height: 200px;
     background-color: #CCC;
 }
 /*HTML 部分代码*/
 <div id="list">
<ul>
<li>家用电器</li>
<li>洗衣机</li>
……
</ul>
</div>
\end{verbatim}
\item 代码含义: \uline{选择 id 为 list 标签,并给标签设置宽度、高度、背景色等样式属性。}
\item 注意事项如下:
\end{itemize}
1) id 是唯一的,同一页面不能出现多个相同的 id 定义。
2) id 名称要求与类选择器相同。
3) 通常当页面中有唯一样式时,采用 id 选择器。
\item 后代选择器与子代选择器
\label{sec:org10256d8}
\begin{enumerate}
\item 后代选择器
\item 写法:选择器 1 选择器 2 选择器 3……\{\} ,每个选择器之间用空格分隔。
\item 代码示例如下:
\begin{verbatim}
     div .li{
         color: yellow;
}
\end{verbatim}

div.li\{\}表示选中的元素包括 div 里面的 class="li"的元素,其中 class="li"的元素可以是
\end{enumerate}
div 的子代,也可以是 div 的后代,也就是孙代及往后。

\begin{enumerate}
\item 子代选择器
\begin{itemize}
\item 写法:选择器 1>选择器 2>选择器 3……\{\} ,每个选择器之间用大于号分隔。
\item 代码示例如下:
\begin{verbatim}
div>ul{
    color: blue;
}
\end{verbatim}
\end{itemize}
\end{enumerate}

div>ul\{\}表示 ul 必须是 div 的直接子代,孙代以后不选中。
\item 交集选择器与并集选择
\label{sec:orge992967}
\begin{enumerate}
\item 交集选择器
\item 写法:选择器1 选择器2……\{\} ,选择器之间没有分隔符。
\item 代码示例如下:
\begin{verbatim}
.list#li{
color: red;
}
\end{verbatim}
\item .list\#li\{\} 元素必须同时具备class="list"并且id="li"样式才能生效。

\item 并集选择器
\begin{itemize}
\item 写法:选择器1,选择器2,……\{\} ,选择器之间用逗号分隔。
\end{itemize}
\item 代码示例如下:
\begin{verbatim}
.li,#li{
    color: red;
    }
\end{verbatim}
\end{enumerate}
div>ul\{\}表示 ul 必须是 div 的直接子代,孙代以后不选中。
\item 交集选择器与并集选择器
\label{sec:org84fa4fa}
\begin{enumerate}
\item 交集选择器
\begin{itemize}
\item 写法:选择器 1 选择器 2……\{\} ,选择器之间没有分隔符。
\item 代码示例如下:
\begin{verbatim}
      .list#li{
          color: red;
}
\end{verbatim}
\end{itemize}
\end{enumerate}
.list\#li\{\} 元素必须同时具备 class="list"并且 id="li"样式才能生效。

\begin{enumerate}
\item 并集选择器
\item 写法:选择器 1,选择器 2,……\{\} ,选择器之间用逗号分隔。
\item 代码示例如下:
\begin{verbatim}
     .li,#li{
         color: red;
}
\end{verbatim}
.li,\#li\{\} 元素只要具备 class="li"或者 id="li",样式即可生效。
\end{enumerate}
\item 伪类选择器
\label{sec:orgbd169d1}
\begin{itemize}
\item 写法:选择器名称:伪类状态\{\}。
\item 代码示例如下:
\begin{verbatim}
   a:hover{
       color: red;
}  
\end{verbatim}
\end{itemize}

常见的伪类状态如下:

\begin{itemize}
\item link:未访问状态。
\item visited:已访问状态。
\item hover:鼠标指向时,即悬停在元素上方时。
\item active:激活选定状态(鼠标点下去没松开)。
\item focus:获得焦点时(input 常用)。
\end{itemize}

超链接多种伪类共存时的顺序如下:link→visited→hover→active。
\item 选择器的命名规则及优先级
\label{sec:org55f2208}
1.选择器的命名规则
    1)只能由字母、数字、下画线组成,不能有其他任何特殊字符。
    2)开头不能是数字,即只能以字母、下画线开头。
2.选择器的优先级
    1)第一原则“近者优先”,最内层选择器永远比外层优先。例如:div ul li > div \#ul,li在 ul 内层,所以 li 标签选择器能覆盖外层 id 选择器。
    2)当作用在同一层时,可以根据选择器优先级权重进行计算。标签选择器优先级为 1,class 选择器优先级为 10,id 选择器优先级为 100。例如:div \#li > div ul .li > div ul li,优先级权重依次为:1+00 > 1+1+10 > 1+1+1。
    3)当优先级权重完全相同时,写在后面的选择器会覆盖前面的选择器。例如:
\begin{verbatim}
div li{ color:red; }
div li{ color:blue; } /* 完全相同的选择器,写在后面的生效 */    
\end{verbatim}
\end{enumerate}
\end{enumerate}
\subsection{第6章 CSS常用属性}
\label{sec:org7280ecc}
\subsubsection{本章学习目标:}
\label{sec:orgff7c307}
\begin{itemize}
\item 熟练掌握CSS常用的文本属性。
\item 熟练掌握CSS常用的背景属性。
\item 正确使用超链接相关的伪类选择器。
\end{itemize}
\subsubsection{CSS常用文本属性}
\label{sec:orga07668b}
\begin{enumerate}
\item 字体、字号与颜色属性
\label{sec:orge487cc8}
\begin{enumerate}
\item 字体
(1)font-family: 设置字体
可以同时设置多个字体,多个字体样式间用逗号分隔,浏览器解析时,会从左往右依次 查找。选择可用字体,当浏览器找不到可用字体时,将使用系统默认字体。

常用的字体:
\begin{center}
\begin{tabular}{ll}
字体名称 & 说明\\
衬线体Serif & 字体在末端有额外的装饰\\
非衬线体Sans-serif(常用)) & 字体再末端没有额外的装饰\\
等宽体Monospace & 所有字符具有相同的宽度,仅针对西文字体\\
\end{tabular}
\end{center}
\end{enumerate}

基本语法如下:
\begin{verbatim}
font-family:Arial, 'Microsoft Yahei', sans-serif;
\end{verbatim}

\begin{enumerate}
\item font-style: 设置字体样式
通常使用其中的两个属性值:
\begin{itemize}
\item 正常(normal)
\item 斜体(italic)。
\end{itemize}

基本语法如下:
\begin{verbatim}
font-style: italic;
\end{verbatim}

\item font: 缩写形式
font 的缩写形式依次为 font-style、font-weight、font-size/line-height、font-family,分别是:
\begin{itemize}
\item 字体样式
\item 字体粗细
\item 字号/行高
\item 字体族。
\end{itemize}

\item 使用font属性的注意事项:
\begin{itemize}
\item 使用时必须严格按照上述顺序。
\item 多个样式之间用空格分隔,且 font-size/line-height 必须作为一对用/分隔。 3)font-size 和 font-family 必须指定,其他样式不指定将采用默认样式显示。 基本语法如下:
\begin{verbatim}
font:italic bold 75%/1.8 'Microsoft Yahei', sans-serif;  
\end{verbatim}
\end{itemize}

\item 字号
(1) font-weight:设置字体粗细
可选属性值:bold 加粗、lighter 细体或者填写 100\textasciitilde{}900 的数字(其中 400 为正常,700 为加粗)。

(2) font-size:设置字体大小
属性值通常为**px 或**\%(其中百分比代表浏览器默认字体大小的百分比,绝大部分浏览器默认为 16px)。

\item 字体颜色
(1) color:设置字体颜色属性值有 3 种表达方式。
\begin{itemize}
\item 直接写颜色的英文名字:red、green、blue 等。
\item 十六进制写法:\#FFFFFF,\#后每位可选值为数字 0\textasciitilde{}9 以及英文的 a\textasciitilde{}f,每两位表示一种颜色,分别对应红绿蓝的比例(最常用,推荐)。
\item rgb 写法:
\begin{itemize}
\item rgb(0\textasciitilde{}255,0\textasciitilde{}255,0\textasciitilde{}255)
\item rgba(0\textasciitilde{}255,0\textasciitilde{}255,0\textasciitilde{}255,0\textasciitilde{}1) 第 4 位数表示透明度,0 表示全透明,1 表示不透明。
\end{itemize}
\end{itemize}

(2)opacity:设置透明度
    属性值为 0\textasciitilde{}1 的数字。

注意:使用 opacity 时当前元素以及子元素均会透明;而使用 rgba 调整时,只会使当前 元素透明,不会改变子元素透明度。

示例代码如下:
\begin{verbatim}
#div1{
    /*使用 rgba 设置 div1 背景透明,则 div1 的子元素不会受影响*/
    /*background-color: rgba(255,0,0,0.5);*/
    /*使用 opacity 设置 div1 透明,则 div1 中的所有背景、文字、子元素均会透明*/
    background-color: red;
    opacity: 0.5;
 }
\end{verbatim}
\end{enumerate}
\item 文本属性
\label{sec:org0510ae4}
\begin{enumerate}
\item line-height
设置行高,属性值表达方式有以下 3 种:
1)像素单位,如 48px。
2)纯数值,表示正常行高的倍数。
3)百分数,表示正常行高的百分数。

line-height 有一个典型应用,就是可以调整元素中文本垂直居中,设置方式为让控件的height等于控件的 line-height

示例如下:
\begin{verbatim}
height:100px;
line-height:100px; /* 设置行高等于高度,则当前元素中文字垂直居中 */
\end{verbatim}

\item text-align
设置块级元素中文字的水平对齐方式,属性值有 left、center、right。

\begin{verbatim}
    <!DOCTYPE html> <html>
    <head>
      <style type="text/css">
      .text_align1{
         height: 30px;
         text-align:left;
          }
      .text_align2{
        height: 30px;
        text-align:center;
         }
      .text_align3{
        height: 30px;
        text-align:right;
         }
     </style>
   </head>
  <body>
    <div class="text_align1">这是文字居左对齐的段落</div> <div class="text_align2">这是文字居中的段落</div> <div class="text_align3">这是文字居右对齐的段落</div>
  </body>
</html>
\end{verbatim}
\item letter-spacing 设置字符间距(以px为单位)
\item text-decoration 文本修饰属性,常用属性值有4个
\begin{itemize}
\item underline 下划线
\item line-through 删除线
\item overline 上划线
\item none 不做修饰

示例如下:
\end{itemize}
\end{enumerate}
\begin{verbatim}
  <!DOCTYPE html>
  <html>
    <head>
      <style type="text/css">
        .text_decoration1{
          text-decoration: overline;
         }
cuyong        .text_decoration2{
          text-decoration: line-through;
        }
       .text_decoration3{
          text-decoration: underline;
       }
     </style>
   </head>
    <body>
      <div class="text_decoration1">这是添加上画线的文字</div>
      <div class="text_decoration2">这是添加删除线的文字</div>
      <div class="text_decoration3">这是添加下画线的文字</div>
    </body>
  </html>
\end{verbatim}
\begin{enumerate}
\item overflow(overflow-x 和 overflow-y)
控制超出范围文本的显示方式,常用属性值有以下三个。
\item auto:根据文字多少自动显示滚动条。
\end{enumerate}
2 )scroll:始终显示滚动条。
3 )hidden:超出范围文本隐藏,可以通过 overflow-x 和 overflow-y 分别设置水平垂直方向的隐藏。

\begin{enumerate}
\item text-overflow
\end{enumerate}
控制超出范围文本的显示方式,常用属性值有以下三个。
\begin{enumerate}
\item auto: 根据文字多少自动显示滚动条。
\item scroll: 始终显示滚动条。
\item hidden: 超出范围文本隐藏,可以通过 overflow-x 和 overflow-y 分别设置水平垂直方向的隐藏。

\item white-space
\end{enumerate}
设置元素内的空白符怎样处理。常见属性值如下:
\begin{enumerate}
\item normal: 默认,空白会被浏览器忽略。
\item nowrap: 设置中文行末不断行显示。
\item pre: 空白会被浏览器保留。 作用类似 HTML 中的 <pre> 标签。
\end{enumerate}

\textbf{重点} 如何让每行多余文字显示省略号?
\begin{enumerate}
\item white-space:nowrap; 如果是中文,需设置行末不断行。
\end{enumerate}
2 )overflow:hidden; 设置控件超出范围隐藏。
3 text-overflow:ellipsis; 设置多余文本省略号显示。

\begin{enumerate}
\item text-shadow
\end{enumerate}
文本阴影,有 4 个属性值。
\begin{enumerate}
\item 水平阴影距离:必写,数值越大,阴影右移。
\item 垂直阴影距离:必写,数值越大,阴影下移。
\item 阴影模糊距离:可写,数值越大,阴影越模糊。默认为 0,不模糊。
\item 阴影颜色:可写,默认为黑色。
\end{enumerate}

示例如下:
\begin{verbatim}
<!DOCTYPE html> <html>
<head>
  <meta charset="UTF-8">
  <title>CSS 样式表</title>
  <style type="text/css">
    .text_shadow{
    text-shadow: 5px 5px 2px red;
    }
  </style>
</head>
<body>
  <h2 class="text_shadow">文字阴影</h2>
</body>
</html>
\end{verbatim}

这里还需要补充,文本阴影可以同时设置多个阴影, 每个阴影效果之间以逗号分隔即可。 例如,将上述阴影代码改为下述语句:
\begin{verbatim}
.text_shadow{
    text-shadow: 5px 3px 3px blue,-5px -3px 3px red;
}
\end{verbatim}

\begin{enumerate}
\item text-indent
\end{enumerate}
首行缩进,可以使用像素值调整段落文字的首行缩进大小。
\begin{verbatim}
text-indent:32px; // 首行缩进 32px,默认字体大小 16px 的情况下,将首行缩进两个字
\end{verbatim}
\end{enumerate}
\subsection{第8章 CSS盒模型与浮动}
\label{sec:orge6c3b0c}
\subsubsection{学习目标}
\label{sec:org10aac07}
\begin{itemize}
\item 了解 CSS 盒模型的基本概念。
\item 熟练掌握盒模型中 margin、border、padding 的使用。
\item 学会使用 CSS3 中关于盒模型的最新属性。
\item 能够使用浮动进行页面布局的调整。
\item 能够使用定位进行页面布局的调整。
\end{itemize}
\subsubsection{盒模型}
\label{sec:org37eb2fc}
\begin{enumerate}
\item 盒模型概述
\label{sec:org0aaebd6}
HTML 文档中的每个元素都可以看作一个盒子,盒模 型规定了这个盒子中的元素内容(content)、内边距(padding)、边框(border)和外边距(margin)所占据的空间。

盒模型结构从内到外依次是 content、padding、border和margin
\begin{enumerate}
\item 盒模型分两种:
\label{sec:orgfb53b66}
\begin{enumerate}
\item 标准盒模型(W3C盒模型)
在标准模式下, 一个元素所占据的总宽度= width + padding(左右)+ border(左右)+ margin(左右), 高度同理。
\item IE盒模型(怪异盒模型)
在怪异盒模型下, 一个块元素的总宽度= width + margin(左右)(即width已经包含了padding和border), 高度同理。
\end{enumerate}
\item 在 CSS 中可以设置 box-sizing 属性来规定使用哪种模型,属性值有两个。
\label{sec:orgc0b9422}
\begin{enumerate}
\item content-box:采用标准模式解析计算,也是默认模式。
\item border-box:采用怪异模式解析计算。
\end{enumerate}
\item 两种模型的区别:
\label{sec:orgcbe1073}
标准盒模型设置的宽度只包含内容区域;  而IE盒模型设置的宽高包含了内容区域 + padding + border。
\end{enumerate}
\item margin : 外边距
\label{sec:orga97528a}
\begin{enumerate}
\item 外边距的属性
\label{sec:org4e54b02}
\begin{itemize}
\item 围绕在元素周围的空白区域就是外边距,外边距是透明的,因此不会遮挡其后面的 元素。

\item 外边距有四个属性可以设置,对应上、下、左、右四个方向,可以使用 margin-top、 margin-bottom、margin-left、margin-right 来分别设置。

\item 简写形式的 margin 可以有 1\textasciitilde{}4 个值:
\begin{itemize}
\item 写一个数值:上、下、左、右四个方向数值相等。
\item 写两个数值:第一个数等于上下外边距,第二个数等于左右外边距。
\item 写三个数值:上、右、下边距,左边默认等于右边。
\item 写四个数值:上、右、下、左 4 个方向的边距。
\end{itemize}
\end{itemize}

实例:
\begin{verbatim}
   <!DOCTYPE html>
   <html>
     <head>
       <style type="text/css">
         #div{
           width: 200px;
           height: 200px;
           color: white; background-color: blue; margin: 0 auto;
         }
         #div .p{
         width: 50px;
         height: 50px;
         color: black; background-color: yellow;
         /*
         margin: 50px 10px 50px 10px;     // 上、右、下、左
         margin: 50px 10px;               // 上=下、右=左
         margin: 50px 10px 50px;          // 上、右、下(左=右) 
        /*父盒子在浏览器中水平居中*/           

        margin: 50px;      // 上=右=下=左
        */
       margin: 0 auto;     /* 设置水平居中 */
    } </style>
   </head>
   <body>
    <div id="div">
      父盒子
      <p class="p">子盒子</p>
    </div>
  </body>
</html>

\end{verbatim}
\item 多个盒子之间的外边距影响
\label{sec:org4956142}
\begin{enumerate}
\item 行内盒子水平排放的外边距
\label{sec:org2ea0b5b}
结论:水平排放的盒子,水平间距是margin的累加。
\item 块级盒子垂直排放的外边距
\label{sec:org87ac60a}
结论:垂直排放的盒子,垂直间距是合并的(取最大值), 外边距小的盒子可能会被覆盖部分内容。 
\item 父、子盒子的垂直外边距合并
\label{sec:orgfeeac89}
在给子盒子添加上外边距后,父、子盒子同时下移,这说明父、 子盒子的外边距合并了。为子盒子添加的上外边距也就是为父盒子添加了上外边距,这对网 页排版造成了一定影响,为了消除这种效果,本书中提供了三种解决方式:
\begin{enumerate}
\item \uline{父盒子添加 overflow:hidden。(常用)}
\item 父盒子添加 padding。
\item 父盒子添加 border。
\end{enumerate}

以上三种方式都可以解决为子盒子添加上外边距后,父盒子也随之移动的情况。但是,实际开发中最常使用第一种方式,由于第二、第三种方式给父盒子添加了不必要的 padding 和 border,可能也会对网页布局细节产生影响,所以不常使用。
\end{enumerate}
\end{enumerate}
\end{enumerate}
\end{document}
